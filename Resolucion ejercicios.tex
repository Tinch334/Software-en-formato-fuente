\documentclass[11pt]{article}

\usepackage[spanish,activeacute]{babel}
\usepackage{titlesec}
\usepackage{graphicx}
\usepackage{subcaption}
\usepackage{float}
\usepackage[bottom]{footmisc}
\usepackage[hidelinks]{hyperref}


\setlength{\parindent}{1.0em}
\setlength{\parskip}{1.0em}
\setlength{\emergencystretch}{5.0em}
\titlespacing*{\section}{0em}{3.5em}{1.5em}
\setcounter{tocdepth}{1}
\renewcommand{\contentsname}{Contenidos}
\hypersetup{
	linktoc=all
}


\title{\Huge Software en formato fuente}
\author{Eugenia Damonte, Ariel Fideleff y Mart\'in Go\~ni}
\date{}


\begin{document}
	\pagenumbering{gobble}
	\maketitle
	\newpage
	\tableofcontents
	\newpage
	\pagenumbering{arabic}
	
	
	\section{Configuraci\'on previa}
		Antes de comenzar a resolver los ejercicios configuramos \texttt{vim} para editar archivos en C. Para hacer esto abrimos el archivo \texttt{\textasciitilde/.vimrc} (en todo caso de no existir, hay que crearlo) que es el archivo de configuraci'on de \texttt{vim}. Estaba vac'io por lo que le le a'nad'imos dos l'ineas: \texttt{set nocp} y \texttt{filetype plugin on}. Lo que hace el primer comando es desactivar el modo de compatibilidad. 'Este hace que algunas de las funciones de \texttt{vim} sean deshabilitadas o modificadas para que se comporte de manera similar a \texttt{vi}, el antecesor de \texttt{vim}. La segunda permite utlizar el plugin \texttt{filetype}.
		
		Luego para asegurarnos de tener todos los paquetes de \texttt{vim} utlizamos el comando \texttt{sudo apt-get install vim-gui-common vim-runtime}. El primer paquete tuvo que instalarse demorando varios minutos por la velocidad de descarga abismal de los repositorios. El segundo, por el otro lado ya estaba instalado en nuestro caso.
		
		Finalmente creamos el archivo de configuraci'on para los archivos con extensi'on \texttt{.c}, llamado \texttt{c.vim}. Para poder crearlo primero tuvimos que crear la carpeta \texttt{\textasciitilde/.vim/ftplugin}, que es donde se ponen los archivos de configuraci'on. Luego abrimos el mismo con \texttt{vim} y escribimos las configuraciones que quer'iamos usar.
		
		\begin{figure}[H]
			\centering
			\begin{subfigure}[b!]{0.7\linewidth}
				\includegraphics[width=\linewidth]{Images/Preamble/Preamble.PNG}
				\caption*{Las configuraciones para los archivos \texttt{.c}.}
			\end{subfigure}
		\end{figure}




















%Remove whitspace when done, for ease of work.
\end{document}